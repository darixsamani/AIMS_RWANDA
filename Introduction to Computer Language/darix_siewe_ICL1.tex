\documentclass{article}
\usepackage{graphicx} % Required for inserting images
\usepackage{verbatim}
\usepackage{listings}
\usepackage{enumitem}
\usepackage[a4paper, margin=25]{geometry}

\usepackage{amsmath}
\title{Introdcuttion to Computer Language Bash Assigment 1}
\author{Darix, SAMANI SIEWE}
\date{\today}


\begin{document}

\maketitle

\begin{enumerate}
    \item Write a bash function that asks for n and computes:
        \begin{enumerate}[label=(\roman*)]
            \item $\displaystyle\sum_{\substack{i=1 \\ i is odd}}^{n} (i+3)$

            \item $\displaystyle\sum_{\substack{i=1 \\ i is even}}^{n} (i+3)$
            
        \end{enumerate}

        The function should have one argument to check if you want to compute for odd or even
increments (i).

        \textbf{Solution: }
        \begin{verbatim}
function question_1 () {
	read -p "enter n: "  n
	s=0
	i=1
	if [ $1 == "odd" ]; then

		while [ $i -le $n ]
		do
			if [ $(($i%2)) -ne 0 ]; then
				s=$((s + i + 3))
			fi
			i=$((i+1))
		done

		echo "The sum S from 1 to $n is: $s "

	
	elif [ $1 == "even" ]; then

		while [ $i -le $n ]
		do
			if [ $(($i%2)) -eq 0 ]; then
				s=$((s + i + 3))
			fi
			i=$((i+1))

		done

		echo "The sum S from 1 to $n is $s" 

	
	else
		echo "you provided a wrong input"
	fi
}
            \end{verbatim}

    \item Write a program that reads a number of the day of the week (from 0 to 6). If it is a
working day, then the program will write the name of the corresponding day. Otherwise,
it will write the word ”Weekend”.

    \textbf{Solution:}

    \begin{verbatim}
function question_2 () {
	read -p "Enter the number of the day to want from 0 to 6: " d

	if [ $d -eq 0 ]; then
		echo "Monday"
	elif [ $d -eq 1 ]; then
		echo "Tuesday"
	elif [ $d -eq 2 ]; then
		echo "Wenesday"
	elif [ $d -eq 3 ]; then
		echo "Thrusday"
	elif [ $d -eq 4 ]; then
		echo "Friday"
	elif [ $d -ge 5 ] && [ $d -le 6 ]; then
		echo "Weekend"
	else
		echo "You provided a wrong input"
	fi
}
    \end{verbatim}

    \item Make a program that, given a month number (from 0 to 11), indicates how many days it
has (28, 30 or 31), ignoring the leap years.

    \textbf{Solution:}

    \begin{verbatim}
function question_3 () {
	read -p "Enter the number of month from 0 to 11: " m

	if [ $m -eq 0 ] || [ $m -eq 2 ] || [ $m -eq 4 ] || [ $m -eq 6 ] || [ $m -eq 7 ] || [ $m -eq 9 ] || [ $m -eq 11 ]; then
		echo "The number of days in $m is : 31"
	elif [ $m -eq 3 ] || [ $m -eq 5 ] || [ $m -eq 8 ] || [ $m -eq 10 ]; then
		echo "The number of days in $m month is 30"
	elif [ $m -eq 1 ]; then 
		echo "The number of days in $m month is 28 or 29"
	else
		echo "You provided the wrong input"
	fi
}
    \end{verbatim}

    \item Write two versions of bash code that displays your name 3 times. One version with a ”for
loop” and another version with a ”while loop”.

    \textbf{Solution}

    \begin{verbatim}
function question_4 () {
	name="Darix SAMANI SIEWE"
	echo "Diplay my name with for loop"
	for i in {1..3};
	do
		echo "$i my name is $name"
	done

	echo "Display my name with while loop"
	i=1

	while [ $i -lt 4 ];
	do
		echo "$i ny name is $name"
		i=$((i+1))
	done
}
    \end{verbatim}

    \item Given two numbers $x_0$ and $x_1$ , the Fibonacci sequence they generate is constructed from the recursion formula $x_{n+1} = x_n + x_{n-1}$ . Calculate the first 15 terms of a Fibonacci sequence, asking the user the initial values.

    \begin{verbatim}
function question_5 () {
	read -p "Enter the first value of fibonnaci X0: " a
	read -p "Enter the second vaue of fibonnaci x1: " b
	fib=0
	x=0
	for i in  {1..15};
	do
		fib=$((a+b))
		echo "the $i terms of fibonnaci is : $fib"
		a=$b
		b=$fib
	done
}
    \end{verbatim}

    \item Compute the first 30 terms of the sequence demode by $x_{n+2} = x_{n+1} - 3x_n$ sequence, given 2 any two initial values $x_0$ and $x_1$

    \begin{verbatim}
function question_6 () {
	read -p "Enter the initiale value of X0: " a
	read -p "Enter the initiale value of X1: " b
	
	for i in {1..30};
	do
		s=$(echo "scale=4; 0.5*$b-3*$a" | bc)
		echo "The $i term is: $s"
		a=$b
		b=$s
	done
}
    \end{verbatim}

    \item \begin{enumerate}
        \item Write bash code that computes the sum of the square of the first n natural numbers.
One version with a ”for loop” and another version with a ”while loop”.
        
        \textbf{Solution:}

        \begin{verbatim}
function question_7_a () {
	read -p "Enter the value of n: " n
	echo "using for loop"
	s=0
	for i in $(seq 1 $n);
	do
		s=$((s+i**2))
	done
	echo "the somme of the square of the first $n naturels numbers is: $s"
	s=0
	echo " using while loop"
	i=0
	while [ $i -le $n ]
	do
		s=$((s+i**2))
		i=$((i+1))
	done
	echo "The somme of the square of the first $n naturels numbers is: $s"

}
        \end{verbatim}

        \item Write bash code that prints, for a natural number M , the smallest natural number
n such that $1_2 + 2_2 + \ldots + n_2 \geq M$ . One version with a ”for loop” and another
version with a ”while loop”.

        \textbf{Solution:}

        \begin{verbatim}
function question_7_b () {
	read -p "Enter the value of M: " M

	echo "Version with for loop"
	n=0
	s=0
	for (( i=0 ; ; i++ ));
	do
		s=$((s+i**2))
		if [ $s -ge $M ]; then
			n=$i
			break
		fi
	done

	echo "The value of n using for loop is : $n"


	echo "Version with while loop"
	s=0
	i=0
	n=0
	while true
	do
		s=$((s+i**2))
		if [ $s -ge $M ]; then
			n=$i
			break
		fi
		i=$((i+1))
	done
	
	echo "The value of n using while loop is: $n"

}
        \end{verbatim}
    \end{enumerate}

    \item Read an integer number n, between 0 and 9, and print its multiplication table up to N
where N is another natural number the program reads.

    \textbf{Solution: }

    \begin{verbatim}
function question_8 () {
	read -p "Enter de number between 0 and 9: " n
	read -p "Enter de number between 0 and 9: " N
	for i in $(seq 1 $N)
	do
		echo " $n X $i = $((n*i))"
	done
}
    \end{verbatim}

    \item Write a program that prints all the numbers between 0 and 40 that are multiples of 3, 7
or 11.

    \textbf{Solution:}
    
    \begin{verbatim}
function question_9 () {
	echo "print all numbers from 0 to 40 that are multiples of 3, 7 or 11:"
	for i in {3..40}
	do
		if [ $(($i%3)) -eq 0 ] || [ $(($i%7)) -eq 0 ] || [ $(($i%11)) -eq 0 ]; then
			echo $i
		fi
	done
}
    \end{verbatim}

    \item Ask for an integer number between 0 and 9, denoted x. Once the user has entered a
correct number (that is one in the range [0, 10)) the program asks for a second integer
number between 0 and 255, denoted max. The program continues asking for the number
until it is correct (that it is in the range [0, 256)). When this is done, show all multiples
of x that are between 0 and max. Then, ask the user whether he/she wants to continue;
if in the affirmative, ask for another couple of numbers; otherwise, finish.

    \textbf{Solution: }

    \begin{verbatim}
function question_10 () {
	read -p "Enter a integer between 0 and 9: " x
	while [ $x -lt 0 ] || [ $x -gt 9 ]
	do
		read -p "Please you need to enter a number between 0 and 9 incluse: " x
	done
	
	read -p "Enter a integer between 0 and 255: " max
	while [ $max -lt 0 ] || [ $max -gt 255 ]
	do
		read -p "Please you need to enter a number between 0 and 255 incluse: " max
	done

	echo "All mutilples of $x betwwen 0 and $max is: "
	for i in $(seq 1 $max)
	do
		if [ $(($i%x)) -eq 0 ]; then
			echo $i
		fi
	done

}
    \end{verbatim}
        
\end{enumerate}

\end{document}
